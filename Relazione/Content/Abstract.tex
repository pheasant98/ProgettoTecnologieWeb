\section{Abstract}
\label{abstract}
\textit{TecArt} è un museo d'arti figurative specializzato in dipinti e sculture di autori italiani ed europei. Il sito sviluppato offre al visitatore la possibilità di consultare agevolmente l'elenco di alcune tra le opere più significative esposte nella sede del museo: in pochi click l'utente può consultare la scheda completa di pezzi prestigiosissimi. È possibile anche restare aggiornati sugli eventi artistici che saranno ospitati dal museo. Le opere sono classificate in \textit{dipinti} e \textit{sculture}, gli eventi in \textit{mostre} e \textit{conferenze}.

Oltre ad un breve estratto della storia del museo e alle indicazioni per contattare o raggiungere la sede, il sito raccoglie le recensioni lasciate dai visitatori, per garantire la massima trasparenza verso gli utenti e mostrare i feedback ricevuti sul servizio offerto.

Infine, in qualunque momento il visitatore può effettuare cercare il contenuto desiderato tramite la barra di ricerca messa a disposizione da tutte le pagine del sito.

Qualora si verificasse un errore, l'utente viene mandato su una pagina di recupero, che gli permette di tornare alla \textit{Home Page}, oltre che di navigare attraverso il menu o usufruire della funzionalità di ricerca.

L'utente affezionato, o semplicemente soddisfatto dell'esperienza nel museo, ha la possibilità di registrarsi e crare un account nel sito tramite la compilazione di un semplice form: in questo modo potrà rilasciare egli stesso delle recensioni, modificarle o rimuoverle.

L'utente amministratore può aggiungere, modificare o rimuovere le informazioni relative agli eventi e alle opere, inoltre può cancellare gli account degli utenti indesiderati e rimuovere le recensioni ritenute inappropriate.

Ogni aspetto del sito è stato pensato per garantire una navigazione semplice e gradevole alla più vasta utenza possibile, prestando particolare attenzione alla scelta della struttura, del \textit{layout}, dei colori e di tutti gli aspetti legati all'accessibilità.
\section{Test}
\label{test}
L'attività conclusiva del progetto è stata la verifica del funzionamento del sito. Il testing si è concentrato su due aspetti, spesso strettamente legati: 
\begin{itemize}
	\item funzionamento del sito e delle interazioni col database, efficacia dei controlli dinamici, efficienza del codice;
	\item esperienza e qualità della navigazione, compatibilità con le varie tecnologie, accessibilità e usabilità.
\end{itemize}
Quando possibile, nuove funzionalità e controlli dinamici sono stati testati parallelamente alla loro implementazione, in modo da rendere più rapida la rilevazione degli errori e la loro correzione. Tuttavia, ciò non è stato sempre possibile, quindi buona parte dell'attività di test è stata svolta al termine della codifica.

\subsection{Strumenti}
\label{test-strumenti}

\subsubsection{Strumenti sviluppatore}
\label{test-strumenti-sviluppatore}
Durante i test di funzionamento del sito sono stati usati gli strumenti sviluppatore messi a disposizione dai browser \textit{Mozilla Fireforx}, \textit{Safari}, \textit{Google Chrome}, \textit{Microsoft Edge}. Si sono rivelati molto utili soprattutto al momento della verifica delle pagine dinamiche, in particolare quelle coinvolte nei controlli JavaScript, nelle quali eventuali bug potevano derivare da combinazioni diverse di parti di codice assemblate al momento.

Sono state usate anche le funzionalità di disabilitazione delle immagini o dei colori, in particolare quelli di \textit{Mozilla Fireforx}.


\subsubsection{Lighthouse}
\label{test-strumenti-lighthouse}
\textit{Lighthouse} è uno strumento \textit{open source} automatizzato, eseguibile su qualsiasi pagina web. Offre \textit{audit} per prestazioni, accessibilità, \textit{progressive web app} (non di nostro interesse), \textit{Search Engine Optimization} (SEO) e altro.

I controlli possono essere eseguiti direttamente dagli strumenti sviluppatore offerti da \textit{Google Chrome} o da riga di comando. Dato in input l'URL della pagina, \textit{Lightouse} esegue i controlli e genera un resoconto, contenente consigli migliorativi. 


\subsubsection{XAMPP}
\label{test-strumenti-xampp-mamp}
Link per il download: \url{https://www.apachefriends.org/it/download.html}

XAMPP è un software \textit{crossplatform} libero costituita da Apache HTTP Server, il database MariaDB e tutti gli strumenti necessari per l'uso i linguaggi di programmazione PHP e Perl. Il nome è un acronimo dei software sopra citati. Su queste piattaforme si appoggia il sito di \textit{TecArt} per la gestione della parte di database e interazioni con esso.

Questo strumento è stato fondamentale per la verifica del funzionamento del sito perché ha permesso di testare tutte le interazioni con database e i vari controlli in PHP.


\subsubsection{Browsershots}
\label{test-strumenti-browsershots}
Link: \url{http://browsershots.org}. 

Browsershots è un'applicazione web che simula un sito web su diversi sistemi operativi e browser. Il servizio è gratuito e permette di verificare il funzionamento delle pagine su un vasto numero di piattaforme diverse, per valutarne compatibilità, funzionamento e accessibilità. I browser di interesse per il sito sono \textit{Mozialla Firefox}, \textit{Google Chrome}, \textit{Safari}, \textit{Microsoft Edge}, \textit{Opera} e \textit{Internet Explorer} 9.

\subsubsection{WAVE}
\label{test-strumenti-wave}
Link: \url{https://wave.webaim.org}.

\textit{Web Accessibility Evaluation tool} (WAVE) è una suite di strumenti di valutazione che permette di migliorere l'accessibilità dei contenuti web. È in grado di identificare molti errori relativi all'accessibilità e allo standard WCAG, ma facilita anche la valutazione umana dei contenuti web.

È possibile usare WAVE online inserendo l'URL di una pagina nel campo apposito. Le estensioni WAVE di \textit{Mozilla Firefox} e \textit{Google Chrome} sono disponibili per testare l'accessibilità direttamente nel browser, utile per controllare pagine protette da password o altamente dinamiche. 


\subsubsection{Silktide}
\label{test-strumenti-silktide}
È un \textit{plugin} gratuito di \textit{Google Chrome} che permette valutare l'accessibilità del sito. Fornisce un resoconto generale del livello di accessibilità, valutando la conformità con lo standard WCAG 2.1. Consente inoltre di navigare il sito simulando le condizioni di utenti affetti da diversi tipi di disabilità, sia visive che comportamentali. 


\subsubsection{NVDA}
\label{test-strumenti-nvda}
Link per il download: \url{https://www.nvda.it/download}.

\textit{Non Visual Desktop Access} (NVDA) è un software \textit{open source} da utilizzabile nei sistemi operativi \textit{Windows7}, \textit{Windows8} e \textit{Windows10}.
Gli utenti affetti da disabilità visive, mediante questo strumento, possono usare un computer in completa autonomia. NVDA è un prodotto gratuito e permette all'utente di conoscere ciò che avviene sullo schermo tramite grazie ad una sintesi vocale.

\subsubsection{Links2}
\label{test-strumenti-links2}
Link per il download: \url{http://links.twibright.com/download.php}.

\textit{Links} è un browser unicamente testuale da linea di comando: permette di navigare le pagine web fruendole unicamente come testo, senza alcuna interfaccia grafica. Questo strumento simula l'impossibilità di accedere alle agevolazioni date dalla grafica, oltre a rappresentare tutta l'utenza che effettivamente non ha gli strumenti per accedere al web in modo diverso dal testo.


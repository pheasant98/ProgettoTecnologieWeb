\section{Test}
\label{test}
L'attività conclusiva del progetto è stata la verifica del funzionamento del sito. Il testing si è concentrato su due aspetti, spesso strettamente legati: 
\begin{itemize}
	\item funzionamento del sito e delle interazioni col database, efficacia dei controlli dinamici, efficienza del codice;
	\item esperienza e qualità della navigazione, compatibilità con le varie tecnologie, accessibilità e usabilità.
\end{itemize}
Quando possibile, nuove funzionalità e controlli dinamici sono stati testati parallelamente alla loro implementazione, in modo da rendere più rapida la rilevazione degli errori e la loro correzione. Tuttavia, ciò non è stato sempre possibile, quindi buona parte dell'attività di test è stata svolta al termine della codifica.

\subsection{Strumenti}
\label{test-strumenti}

\subsubsection{Strumenti sviluppatore}
\label{test-strumenti-sviluppatore}


\subsubsection{Lighthouse - \textit{Google Chrome audits}}
\label{test-strumenti-lighthouse}
Lighthouse è uno strumento \textit{open source} automatizzato, eseguibile su qualsiasi pagina web. Offre \textit{audit} per prestazioni, accessibilità, \textit{progressive web app} (non di nostro interesse), \textit{Search Engine Optimization} (SEO) e altro.

I controlli possono essere eseguiti direttamente dagli strumenti sviluppatore offerti da \textit{Google Chrome} o da riga di comando. Dato in input l'URL della pagina, \textit{Lightouse} esegue i controlli e genera un resoconto, contenente consigli migliorativi. 


\subsubsection{XAMPP}
\label{test-strumenti-xampp}


\subsubsection{Browsershots}
\label{test-strumenti-browsershots}
Link: \url{http://browsershots.org}. 

Browsershots è un'applicazione web che simula un sito web su diversi sistemi operativi e browser. Il servizio è gratuito e permette di verificare il funzionamento delle pagine su un vasto numero di piattaforme diverse, per valutarne compatibilità, funzionamento e accessibilità.

\subsubsection{WAVE}
\label{test-strumenti-wave}


\subsubsection{Silktide}
\label{test-strumenti-silktide}
È un \textit{plugin} gratuito di \textit{Google Chrome} che permette valutare l'accessibilità del sito. Fornisce un resoconto generale del livello di accessibilità, valutando la conformità con lo standard WCAG 2.1. Consente inoltre di navigare il sito simulando le condizioni di utenti affetti da diversi tipi di disabilità, sia visive che comportamentali. 


\subsubsection{NVDA}
\label{test-strumenti-nvda}
Link per il download: \url{https://www.nvda.it/download}.




\subsubsection{Links2}
\label{test-strumenti-links2}
Link per il download: \url{http://links.twibright.com/download.php}.


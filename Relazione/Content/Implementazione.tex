\section{Implementazione}
\label{implementazione}

\subsection{XHTML}
\label{implementazione-html}
Per la struttura statica delle pagine del sito è stato utilizzato XHTML 1.0 Strict. Si è scelto questo linguaggio perché supportato da gran parte dei browser, anche nelle loro versioni meno recenti, e perché regolato da una sintassi molto rigida, che rende quasi univoca l'interpretazione dei diversi elementi da parte dei browser stessi. È stato scartato HTML5 perché tutte le funzionalità necessarie allo sviluppo del sito erano già offerte da XHTML 1.0 Strict e attualmente non è ancora completamente supportato da tutti i browser per i quali si vuole rendere compatibile il sito.

Il contenuto statico delle pagine è composto da header, breadcrumbs (nella struttura, non sempre nel contenuto), struttura del content e footer.

Per garantire la separazione tra contenuto e presentazione non sono stati usati tag XHTML 1.0 Strict di stile nè è stato inserito codice CSS all'interno dei file \textit{.html}. Si è inoltre evitato di inserire tag XHTML 1.0 Strict strutturali col solo scopo di usarli come base per le regole CSS. 

\subsubsection{Strumenti}
\label{implementazione-html-strumenti}

\paragraph{w3schools HTML reference}
\label{implementazione-html-strumenti-w3schools-reference}
Per valutare la compatibilità dei tag utilizzati rispetto ai vari browser, anche nelle loro versioni meno recenti, ci si è affidati al sito di \textit{w3schools}, che riporta, per ogni tag XHTML 1.0 Strict, le informazioni di compatibilità.


\subsection{CSS}
\label{implementazione-css}
Per la presentazione grafica del sito è stato utilizzato CSS3. Nella scelta delle regole CSS si è cercato di garantire l'accessibilità ad utenti affetti da disabilità visive non completamente invalidanti e di evitare effetti grafici che avrebbero potuto causare disagio ad utenti affetti da disturbi comportamentali.

\subsubsection{Strumenti}
\label{implementazione-css-strumenti}

\paragraph{w3schools CSS reference}
\label{implementazione-css-strumenti-w3schools-reference}
Per valutare la compatibilità delle regole utilizzate rispetto ai vari browser, anche nelle loro versioni meno recenti, ci si è affidati al sito di \textit{w3schools}, che riporta, per ogni regola CSS, le informazioni di compatibilità.

\subsection{PHP}
\label{implementazione-php}
Per i controlli sull'input lato server e le interazioni col database è stato usato PHP7. Nonostante tutti i controlli lato client siano fatti in linguaggio JavaScript, è necessario effettuarli anche lato server qualora i primi falliscano, ad esempio se JavaScript è disabilitato. L'uso di PHP ha permesso anche di diversificare il comportamento dinamico delle pagine, a seconda dei risultati ottenuti dalle operazioni col database e del tipo di utente (generico, autenticato o amministratore) che interagisce col sito. In questo modo, infatti, si ottiene la completa separazione del comportamento dal contenuto e dalla presentazione.

Il file \textit{.php} sono stati divisi in \textit{Database}, \textit{Repository}, \textit{Controller}, \textit{Utilities} e [NomePagina].php. 
\begin{itemize}
	\item \textbf{Database:} un singolo file chiamato DatabaseAccess.php offre metodi per la preparazione e l'esecuzione di query al database differenziando quelle che fanno uso dello select statement da quelle prive di esso;
	\item \textbf{\textit{Repository}:} uno per ogni tabella del database, offre metodi per l'esecuzione di query sulla rispettiva tabella, che implementano le operazioni \textit{CRUD:} inserimento, lettura, modifica e cancellazione. Di queste sono fornite numerose versioni al fine di implementare le funzionalità in modo soddisfacente. Per la tabelle Opere sono offerti due tipi di contenuto diversi, dipinti e sculture, che però, a differenza delle due tipologie di evento, necessitano di campi dati distinti. Dunque si sono costruite query diversificate per risolvere questa evenienza;
	\item \textbf{\textit{Controller}:} invocano i metodi esposti dalle \textit{Repository} mettendo in comunicazione le pagine [NomePagina].php col database.
	In particolare forniscono un'interfaccia per trasformare i dati dal formato predisposto all'utente, pagine [NomePagina].php, a quello utilizzato dal database, \textit{Repository}, e viceversa. Offrono inoltre tutti i metodi per eseguire i controlli sull'input dell'utente: alcuni controlli generici (trim per l'eliminazione degli spazi bianchi in testa ed in coda, sostituzione dei caratteri speciali con le apposite \textit{entity}, eliminazione dei possibili tag XHTML o comandi PHP e gestione delle parole in lingua straniera con opportuni tag come [en][/en]) sono comuni a tutti i controller, altri invece sono specifici del contenuto che gestiscono.
	\item \textbf{\textit{Utilities}:} il file DateUtilities.php gestisce le conversioni del formato di data; il file FileUtilities.php effettua i controlli sul caricamento dei file da parte dell'utente, poiché l'inserimento di un'opera nel sito richiede anche il caricamento di un'immagine; mentre DataUtilities.php contiene solo funzioni statiche, FileUtilities.php ne è composto solo in parte;
	\item \textbf{[NomePagina].php:} una per ogni pagina \textit{.html} statica, gestisce il caricamento dei contenuti dinamici del sito e la loro organizzazione nella pagina, la gestione delle connessioni al database, la chiamata delle funzioni per l'interazione tra pagina e database delegando il compito controller e le funzionalità a disposizione a seconda del tipo di utente che sta navigando il sito.
\end{itemize}
Grazie a questa suddivisione il codice risulta molto più leggibile e di facile debug oltre ad evidenziare una separazione tra ciò che viene visualizzato all'utente e i dati che invece sono salvati.

\subsubsection{Sessioni}
\label{implementazione-php-sessioni}
Per memorizzare alcune informazioni nel passaggio da una pagina all'altra sono state usate le sessioni. Più precisamente, si tiene traccia dello username dell'utente e della sua tipologia (admin, autenticato non admin, non autenticato), della paginazione dei contenuti (se si tratta di pagine che coinvolgono gli elenchi di opere, eventi, recensioni o utenti) e altri parametri simili necessari a mantenere le informazioni essenziali ad un'ottimale esperienza d'suo dell'utente durante l'intera navigazione.

Per gestire adeguatamente il caso in cui un utente conservasse il bookmark di una pagina richiedente specifici parametri di sessione per essere acceduta, si è deciso di prevedere un'impostazione di default dei parametri stessi in assenza del passaggio di valori effettivi. In questo modo sarà comunque possibile accedere alla pagina salvata ed interagire con essa e col breadcrumbs senza rischio di errori.

\subsection{JavaScript}
\label{implementazione-javascript}
Il linguaggio JavaScript è stato usato principalmente per i controlli sull'input lato client: nonostante infatti i controlli siano effettuati anche lato server (vedi §\ref{implementazione-php}), poiché le elaborazioni \textit{server side} sono più lente e onerose, a causa della comunicazione di rete, è preferibile effettuare più controlli possibile lato client, e fermare lì eventuali input errati. Per questo motivo le operazioni effettuate dai metodi JavaScript per controllare l'input sono analoghe a quelle dei corrispondenti metodi PHP lato server.

Viene utilizzato un unico script JavaScript contenuto in un unico file in modo da ridurre i tempi di caricamento, questo contiene tutti i metodi per il controllo e la gestione degli input, differenziati per opera, evento, recensione, utente e risultato della ricerca.

Per tutti i form si controlla l'appropriatezza dei dati inseriti dall'utente: se sono errati, sopra il campo di input viene mostrato un messaggio di errore, spesso associato ad un suggerimento sul tipo di errore commesso \footnote{Per agevolare ancora di più l'utente nella compilazione dei form, sono previsti dei suggerimenti associati ai campi a completamento meno intuitvo; questi suggerimenti sono statici, inseriti quindi in XHTML 1.0 Strict.}.

Oltre al controllo degli input e segnalazione degli errori, JavaScript gestisce le seguenti funzionalità:
\begin{itemize}
	\item la comparsa e la scomparsa del menu ad hamburger, per il layout mobile;
	\item la rimozione della mappa dalla pagina dei \textit{Contatti} se questa viene visualizzata tramite browser \textit{Internet Explorer 9} e \textit{Internet Explorer 10}, poiché non supportata; questo non diminuisce la comprensibilità delle informazioni fornite, in quanto la mappa è comunque corredata da indicazioni stradali scritte;
	\item la corretta abilitazione dei campi dati delle opere durante l'inserimento e la modifica, in base alla selezione dello stile (dipinto o scultura);
	\item la corretta composizione dei filtri per la pagina \textit{Gestione contenuti}, in base al contenuto desiderato (opera o evento); 
	\item la visualizzazione dell'anteprima dell'immagine al momento dell'inserimento o della modifica di un'opera.
\end{itemize}


È stato possibile implementare in JavaScript tutti i controlli previsti in PHP, l'unica eccezione è il controllo sulla dimensione dell'immagine caricata tramite browser \textit{Internet Explorer 9}, a causa del mancato sopporto dei metodi necessari; si è comunque avuto cura di raffinare il più possibile la verifica su questo tipo di input, in modo da ridurre al minimo il numero di invii scorretti verso al server (vengono ugualmente controllati il numero dei file selezionati e la loro estensione).

Si è deciso di non adottare JQuery per la risoluzione di questo problema di compatibilità, come per il resto delle funzionalità implementate, per evitare di appesantire inutilmente il sito, trattandosi questo dell'unico caso di controllo incompleto lato client; tutti i restati sono stati correttamente implementati utilizzando esclusivamente JavaScript.


\subsubsection{Strumenti}
\label{implementazione-javascript-strumenti}

\paragraph{MDN web docs}
\label{implementazione-javascript-strumenti-mdn}
Per valutare la compatibilità dei meotodi utilizzati rispetto ai vari browser, anche nelle loro versioni meno recenti, ci si è affidati al sito di \textit{MDN} (\url{https://developer.mozilla.org/it/docs/Web/JavaScript}), che riporta, per ogni metodo, le informazioni di compatibilità.


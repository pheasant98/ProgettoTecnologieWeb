\section{Analisi}
\label{analisi}

\subsection{Categorie di utenti}
\label{analisi-categorie-utenti}
Il museo si rivolge ad un'utenza molto ampia: le opere esposte possono attirare scolaresche, docenti, amatori, specialisti del settore, famiglie, anziani.

Gli utenti specialisti o appassionati d'arte sono interessati al lato tecnico e critico delle opere: le schede descrittive dovranno quindi essere ben curate e l'esposizione delle informazioni chiara e precisa. Famiglie e scolaresche puntano ad un'esperienza meno appronfondita ma più ampia e didattica: la collezione di opere presentate deve quindi essere ampia e variegata, e gli eventi organizzati devono prevedere mostre e cicli di conferenze sul più ampio spettro di correnti e tematiche artistiche possibile.

In generale, il sito deve essere pronto a corrispondere alle esigenze di un'utenza varia e ampia: l'interfaccia deve essere intutiva e leggibile, la terminologia non troppo specialistica, la grafica adatta ad un pubblico di tutte le età, nè troppo seria nè eccessivamente giocosa.

\subsection{Attori principali}
\label{analisi-casi-uso-attori-principali}
Sono state individuate tre categorie principali di attori: \textit{utente generico}, \textit{utente autenticato}, \textit{utente ammininstratore}.

\subsubsection{Utente generico}
\label{analisi-casi-uso-attori-principali-utente-generico}
L'\textit{utente generico} è un visitatore anonimo del sito, che non ha effettuato né registrazione né login; egli ha la possibilità di interagire col sito mediante il form di registrazione, la barra di ricerca e i filtri di visualizzazione di opere ed eventi. L'utente generico può prendere visione di tutti i contenuti senza però poterli modificare, eliminare e senza poterne aggiungere di nuovi.

\subsubsection{Utente autenticato}
\label{analisi-casi-uso-attori-principali-utente-autenticato}
L'\textit{utente autenticato} eredita tutte le funzionalità dell'utente generico. In aggiunta, avendo un account personale nel database, può accedere alla propria area utente, gestire i propri dati personali, lasciare recensioni, modificarle ed eliminarle.

\subsubsection{Utente amministratore}
\label{analisi-casi-uso-attori-principali-utente-amministratore}
L'\textit{utente amministratore} eredita anch'esso le funzionalità di quello generico, cui si aggiungono quelle legate al possesso di un account (visualizzazione dei dati personali e loro modifica, come per l'\textit{utente autenticato}) e  la gestione di tutti i contenuti presenti nel sito. L'amministratore può inserire, modificare e rimuovere le pagine relative ad eventi e opere, e eliminare, se necessario, recensioni o account di utenti indesiderati.


\subsection{Casi d'uso}
\label{analisi-casi-uso}

\subsubsection{Utente generico}
\label{analisi-casi-uso-attori-principali-utente-generico}
I casi d'uso che riguardano questa categoria di utenti sono:
\begin{itemize}
	\item visualizzazione della \textit{Home page} (vedi §\ref{analisi-casi-uso-attori-principali-utente-generico-1});
	\item interazione coi link del \textit{breadcrumb} (vedi §\ref{analisi-casi-uso-attori-principali-utente-generico-2});
	\item interazione coi link del \textit{menu} (vedi §\ref{analisi-casi-uso-attori-principali-utente-generico-3});
	\item visualizzazione della pagina \textit{Opere} (vedi §\ref{analisi-casi-uso-attori-principali-utente-generico-4});
	\item interazione con la pagina \textit{Opere} (vedi §\ref{analisi-casi-uso-attori-principali-utente-generico-5});
	\item visualizzazione della pagina \textit{Eventi} (vedi §\ref{analisi-casi-uso-attori-principali-utente-generico-6});
	\item interazione con la pagina \textit{Eventi} (vedi §\ref{analisi-casi-uso-attori-principali-utente-generico-7});
	\item visualizzazione della pagina \textit{Cosa dicono di noi} (vedi §\ref{analisi-casi-uso-attori-principali-utente-generico-8});
	\item interazione con la pagina \textit{Cosa dicono di noi} (vedi §\ref{analisi-casi-uso-attori-principali-utente-generico-9});
	\item visualizzazione della pagina \textit{Contatti} (vedi §\ref{analisi-casi-uso-attori-principali-utente-generico-10});
	\item visualizzazione della pagina \textit{Login} (vedi §\ref{analisi-casi-uso-attori-principali-utente-generico-11});
	\item esecuzione di una ricerca mediante la barra nell'\textit{header} (vedi §\ref{analisi-casi-uso-attori-principali-utente-generico-12});
	\item visualizzazione risultato della ricerca nella pagina \textit{Ricerca} (vedi §\ref{analisi-casi-uso-attori-principali-utente-generico-13});
	\item interazione con la pagina \textit{Ricerca} (vedi §\ref{analisi-casi-uso-attori-principali-utente-generico-14});
	\item visualizzazione della pagina \textit{Opera} (vedi §\ref{analisi-casi-uso-attori-principali-utente-generico-15});
	\item visualizzazione della pagina \textit{Evento} (vedi §\ref{analisi-casi-uso-attori-principali-utente-generico-16});
	\item visualizzazione della pagina \textit{Recensione} (vedi §\ref{analisi-casi-uso-attori-principali-utente-generico-17});
	\item visualizzazione della pagina \textit{Registrazione} (vedi §\ref{analisi-casi-uso-attori-principali-utente-generico-18});
	\item creazione di un account dalla pagina \textit{Registrazione} (vedi §\ref{analisi-casi-uso-attori-principali-utente-generico-19}).
\end{itemize}

\paragraph{Visualizzazione della \textit{Home page}}
\label{analisi-casi-uso-attori-principali-utente-generico-1}
La \textit{Home page} sarà la prima ad essere visualizzata se accede al sito digitando l'\textit{url} del sito sul browser. In ogni momento sarà possibile tornare alla pagina tramite il menu presente in ogni pagina del sito; se, nel percorso seguito all'interno del sito per arrivare alla pagina corrente, l'utente avrà visitato la \textit{Home page}, questa sarà presente nel \textit{breadcrumb} e accessibile anche tramite esso. La pagina conterrà una breve storia del museo, corredata con una fotografia della sede.

\paragraph{Interazione coi link del \textit{breadcrumb}}
\label{analisi-casi-uso-attori-principali-utente-generico-2}
In ogni momento, durante la navigazione nel sito, l'utente avrà a disposizione la sequenza ordinata di pagine tramite le quali è univocamente possibile raggiungere quella corrente. Tutti i nomi delle pagine attrversate saranno link sensibili, in modo che sia possibile tornare a visitarle direttamente dal breadcrumb, tranne quello della pagina corrente (per evitare link circolari).

paragraph{Interazione coi link del \textit{menu}}
\label{analisi-casi-uso-attori-principali-utente-generico-3}
In ogni momento, durante la navigazione nel sito, l'utente avrà a disposizione un menu per raggiungere tutte le pagine di interesse del sito: \textit{Home page}, \textit{Opere}, \textit{Eventi}, \textit{Cosa dicono di noi} (per le recensioni) e \textit{Contatti}. Tutti i nomi delle pagine in elenco saranno link sensibili, tranne, in caso ci si trovasse in una di esse, quello della pagina corrente (per evitare link circolari).

\paragraph{Visualizzazione della pagina \textit{Opere}}
\label{analisi-casi-uso-attori-principali-utente-generico-4}

\paragraph{Interazione con la pagina \textit{Opere}}
\label{analisi-casi-uso-attori-principali-utente-generico-5}

\paragraph{Visualizzazione della pagina \textit{Eventi}}
\label{analisi-casi-uso-attori-principali-utente-generico-6}

\paragraph{Interazione con la pagina \textit{Eventi}}
\label{analisi-casi-uso-attori-principali-utente-generico-7}

\paragraph{Visualizzazione della pagina \textit{Cosa dicono di noi}}
\label{analisi-casi-uso-attori-principali-utente-generico-8}

\paragraph{Interazione con la pagina \textit{Cosa dicono di noi}}
\label{analisi-casi-uso-attori-principali-utente-generico-9}

\paragraph{Visualizzazione della pagina \textit{Contatti}}
\label{analisi-casi-uso-attori-principali-utente-generico-10}

\paragraph{Visualizzazione della pagina \textit{Login}}
\label{analisi-casi-uso-attori-principali-utente-generico-11}

\paragraph{Esecuzione di una ricerca mediante la barra nell'\textit{header}}
\label{analisi-casi-uso-attori-principali-utente-generico-12}

\paragraph{Visualizzazione risultato della ricerca nella pagina \textit{Ricerca}}
\label{analisi-casi-uso-attori-principali-utente-generico-13}

\paragraph{Interazione con la pagina \textit{Ricerca}}
\label{analisi-casi-uso-attori-principali-utente-generico-14}

\paragraph{Visualizzazione della pagina \textit{Opera}}
\label{analisi-casi-uso-attori-principali-utente-generico-15}

\paragraph{Visualizzazione della pagina \textit{Evento}}
\label{analisi-casi-uso-attori-principali-utente-generico-16}

\paragraph{Visualizzazione della pagina \textit{recensione}}
\label{analisi-casi-uso-attori-principali-utente-generico-17}

\paragraph{Visualizzazione della pagina \textit{Registrazione}}
\label{analisi-casi-uso-attori-principali-utente-generico-18}

\paragraph{Creazione di un account dalla pagina \textit{Registrazione}}
\label{analisi-casi-uso-attori-principali-utente-generico-19}

\subsubsection{Utente autenticato}
\label{analisi-casi-uso-attori-principali-utente-autenticato}
Oltre a tutti i casi d'uso esposti per l'utente generico, riguardano l'utente autenticato anche:
\begin{itemize}
	\item autenticazione dalla pagina \textit{Login} (vedi §\ref{analisi-casi-uso-attori-principali-utente-autenticato-1});
	\item visualizzazione della pagina \textit{Utente} (vedi §\ref{analisi-casi-uso-attori-principali-utente-autenticato-2});
	\item inserimento di una recensione dalla pagina \textit{Lascia una recensione} (vedi §\ref{analisi-casi-uso-attori-principali-utente-autenticato-3});
	\item visualizzazione delle proprie recensioni dalla pagina \textit{Gestione recensioni} (vedi §\ref{analisi-casi-uso-attori-principali-utente-autenticato-4});
	\item modifica di una recensione dalla pagina \textit{Gestione recensioni} (vedi §\ref{analisi-casi-uso-attori-principali-utente-autenticato-5});
	\item visualizzazione della pagina \textit{Recensione modificata} (vedi §\ref{analisi-casi-uso-attori-principali-utente-autenticato-6});
	\item rimozione di una recensione dalla pagina \textit{Gestione recensioni} (vedi §\ref{analisi-casi-uso-attori-principali-utente-autenticato-7});
	\item modifica dei propri dati utente dalla pagina \textit{Modifica dati utente} (vedi §\ref{analisi-casi-uso-attori-principali-utente-autenticato-8});
	\item visualizzazione della pagina \textit{Dati utente modificati} (vedi §\ref{analisi-casi-uso-attori-principali-utente-autenticato-9});
	\item logout dal proprio account (vedi §\ref{analisi-casi-uso-attori-principali-utente-autenticato-10}).
\end{itemize}

\paragraph{Autenticazione dalla pagina \textit{Login}}
\label{analisi-casi-uso-attori-principali-utente-autenticato-1}

\paragraph{Visualizzazione della pagina \textit{Utente}}
\label{analisi-casi-uso-attori-principali-utente-autenticato-2}

\paragraph{Inserimento di una recensione dalla pagina \textit{Lascia una recensione}}
\label{analisi-casi-uso-attori-principali-utente-autenticato-3}

\paragraph{Visualizzazione delle proprie recensioni dalla pagina \textit{Gestione recensioni}}
\label{analisi-casi-uso-attori-principali-utente-autenticato-4}

\paragraph{Modifica di una recensione dalla pagina \textit{Gestione recensioni}}
\label{analisi-casi-uso-attori-principali-utente-autenticato-5}

\paragraph{Visualizzazione della pagina \textit{Recensione modificata}}
\label{analisi-casi-uso-attori-principali-utente-autenticato-6}

\paragraph{Rimozione di una recensione dalla pagina \textit{Gestione recensioni}}
\label{analisi-casi-uso-attori-principali-utente-autenticato-7}

\paragraph{Modifica dei propri dati utente dalla pagina \textit{Modifica dati utente}}
\label{analisi-casi-uso-attori-principali-utente-autenticato-8}

\paragraph{Visualizzazione della pagina \textit{Dati utente modificati}}
\label{analisi-casi-uso-attori-principali-utente-autenticato-9}
Se la modifica dei dati personali (vedi §\ref{analisi-casi-uso-attori-principali-utente-autenticato-8}) è andata a buon fine, l'utente viene indirizzato automaticamente a questa pagina, che conferma la modifica appena effettuata.

\paragraph{Logout dal proprio account}
\label{analisi-casi-uso-attori-principali-utente-autenticato-10}
L'utente amministratore cliccherà il button \texttt{Logout} posto nell'\textit{header} disconnettendosi dal proprio account: da questo momento navigherà il sito come utente generico. Questa operazione può essere effettuata da qualunque pagina del sito.

\subsubsection{Utente amministratore}
\label{analisi-casi-uso-attori-principali-utente-amministratore}
Oltre a tutti i casi d'uso esposti per l'utente generico, riguardano l'utente amministratore anche:
\begin{itemize}
	\item autenticazione dalla pagina \textit{Login} (vedi §\ref{analisi-casi-uso-attori-principali-utente-amministratore-1});
	\item visualizzazione della pagina \textit{Utente} (vedi §\ref{analisi-casi-uso-attori-principali-utente-amministratore-2});
	\item visualizzazione della pagina \textit{Gestione contenuti} (vedi §\ref{analisi-casi-uso-attori-principali-utente-amministratore-3});
	\item interazione con la pagina \textit{Gestione dei contenuti} (vedi §\ref{analisi-casi-uso-attori-principali-utente-amministratore-4});
	\item modifica dei dati di un'opera dalla pagina \textit{Modifica opera} (vedi §\ref{analisi-casi-uso-attori-principali-utente-amministratore-5});
	\item modifica dei dati di un evento dalla pagina \textit{Modifica evento} (vedi §\ref{analisi-casi-uso-attori-principali-utente-amministratore-6});
	\item rimozione di un evento o un'opera dal sito dalla pagina \textit{Gestione contenuti} (vedi §\ref{analisi-casi-uso-attori-principali-utente-amministratore-7});
	\item rimozione di una recensione dal sito dalla pagina \textit{Gestione recensioni} (vedi §\ref{analisi-casi-uso-attori-principali-utente-amministratore-8});
	\item rimozione di un utente dal sito dalla pagina \textit{Gestione utenti} (vedi §\ref{analisi-casi-uso-attori-principali-utente-amministratore-9});
	\item modifica dei propri dati utente dalla pagina \textit{Modifica dati utente} (vedi §\ref{analisi-casi-uso-attori-principali-utente-amministratore-10});
	\item visualizzazione della pagina \textit{Dati utente modificati} (vedi §\ref{analisi-casi-uso-attori-principali-utente-amministratore-11});
	\item logout dal proprio account (vedi §\ref{analisi-casi-uso-attori-principali-utente-amministratore-12}).
\end{itemize}

\paragraph{Autenticazione dalla pagina \textit{Login}}
\label{analisi-casi-uso-attori-principali-utente-amministratore-1}

\paragraph{Visualizzazione della pagina \textit{Utente}}
\label{analisi-casi-uso-attori-principali-utente-amministratore-2}

\paragraph{Visualizzazione della pagina \textit{Gestione contenuti}}
\label{analisi-casi-uso-attori-principali-utente-amministratore-3}

\paragraph{Interazione con la pagina \textit{Gestione dei contenuti}}
\label{analisi-casi-uso-attori-principali-utente-amministratore-4}

\paragraph{Modifica dei dati di un'opera dalla pagina \textit{Modifica opera}}
\label{analisi-casi-uso-attori-principali-utente-amministratore-5}

\paragraph{Modifica dei dati di un evento dalla pagina \textit{Modifica evento}}
\label{analisi-casi-uso-attori-principali-utente-amministratore-6}

\paragraph{Rimozione di un evento o un'opera dal sito dalla pagina \textit{Gestione contenuti}}
\label{analisi-casi-uso-attori-principali-utente-amministratore-7}

\paragraph{Rimozione di una recensione dal sito dalla pagina \textit{Gestione recensioni}}
\label{analisi-casi-uso-attori-principali-utente-amministratore-8}

\paragraph{Rimozione di un utente dal sito dalla pagina \textit{Gestione utenti}}
\label{analisi-casi-uso-attori-principali-utente-amministratore-9}

\paragraph{Modifica dei propri dati utente dalla pagina \textit{Modifica dati utente}}
\label{analisi-casi-uso-attori-principali-utente-amministratore-10}
Dalla pagina personale

\paragraph{Visualizzazione della pagina \textit{Dati utente modificati}}
\label{analisi-casi-uso-attori-principali-utente-amministratore-11}
Se la modifica dei dati personali (vedi §\ref{analisi-casi-uso-attori-principali-utente-amministratore-8}) è andata a buon fine, l'utente viene indirizzato automaticamente a questa pagina, che conferma la modifica appena effettuata.

\paragraph{Logout dal proprio account}
\label{analisi-casi-uso-attori-principali-utente-amministratore-12}
L'utente amministratore cliccherà il button \texttt{Logout} posto nell'\textit{header} disconnettendosi dal proprio account: da questo momento navigherà il sito come utente generico. Questa operazione può essere effettuata da qualunque pagina del sito.
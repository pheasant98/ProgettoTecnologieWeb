% File: template.tex
% InputFile: packages.tex
\documentclass[11pt,a4paper]{article}
% Packages per la formattazione
%package per le sezioni a quattro indici
\usepackage{titlesec} 
\usepackage[left=3.5cm,right=3.5cm,top=3.1cm,bottom=2.6cm]{geometry}
\usepackage[italian]{babel}
\usepackage[utf8]{inputenc}
\usepackage{lastpage}
\usepackage[T1]{fontenc}
\usepackage{geometry}
%package per la gestione delle immagini
\usepackage{graphicx}
%pakage per ridefinire font caption
\usepackage[font=small,labelfont=bf]{caption} 
\geometry{a4paper}
%package di style per  l/c/rhead l/rfoot
\usepackage{fancyhdr} 

%package per i punti nella tableofcontents
\usepackage{tocloft} 
\usepackage{longtable}
\pagestyle{fancy}

% package per i colori
\usepackage{color} 
\usepackage[table]{xcolor} 
\definecolor{acqua}{rgb}{0.0, 1.0, 1.0}

\usepackage{hyperref}
\hypersetup {
	colorlinks=true,
	linkcolor=black,
	urlcolor=linkcolor
}		 

%package meno complesso per la gestione delle tabelle
\usepackage{tabularx}
\usepackage{pdfpages}
%package per il simbolo dell'euro
\usepackage{eurosym}
%package per unire 2 righe
\usepackage{multirow}
%package per mettere il foglio in orizzontale 
 \usepackage{pdflscape}
\usepackage{titlesec}
\usepackage{epstopdf}
\usepackage{makecell}
	
% Indentazione prima linea dei paragrafi
\usepackage{indentfirst}

%indice con i puntini
\usepackage{tocloft}
\renewcommand\cftsecleader{\cftdotfill{\cftdotsep}}